\chapter{Conclusions} % Main chapter title

\label{Chapter5}
\setlength{\parindent}{0.5cm} The aim of this master's thesis was to investigate the solidification phenomena through different models. The enthalpy-porosity model is widely used for phase transition processes and the Lee model using a newly proposed way of calculating the nucleation rate so it can shape the physical phenomena accounting in such processes. On the last part of this thesis, it is proposed a multiregion solver which permits conjugate heat transfer between solid and liquid zones. The main difference with respect to the already offered capabilities on OpenFOAM is the possibility of allowing a fluid region that can hande with phase-changes. 

\noindent In the former part of this master's thesis, the natural convection heat transfer mechanism is studied. To do so, a first analysis using BuoyantBoussinesqPimpleFoam, a native pure convection solver which deals with both laminar and turbulent unsteady heat transfer using the Boussinesq approximation is used. Results on this solver depict a quasi-linear pattern on the temperature distribution and only one convection current is observed in the middle of the computational domain. A polynomial density variation for the water equation of state and a new expression regarding the gravity terms of the momentum equation are implemented on the basis of the native solver. The results obtained on this solver showed a completely different pattern. The density inversion point, which is tipically observed in experimental works due to the own phenomena's behavior, is similarly observed in the simulations. Two convection currents can be appreciated in the cavity. These flows recirculate oppositely, they counteract with each other, being the upper flow, the more dense. A small change in density, in this case 0.17 $kg.m^-3$, evidences this change in the flow pattern's behavior. Numerical results are, in overall, in good agreement with the results found in the literature. All the evaluated magnitudes nearly overlapped the data of the bibliography. Maximum differences with respect to Fluent numerical results are found to be below 16\% for the V-velocity component along the vertical mid-plane. At time equal to 1500s, the solution reached a quasi-steady solution which is used as initial solution in the solidification process. The fact of finding the propper behavior of the convective fluid is of crucial importance in the way of which the ice layer on the phase transition process will evolve. 

\noindent In the second part of the thesis, water solidification process is investigated by means of a theoretical model, enthalpy-porosity and a semi-empirical one, Lee model. Within a multiphase, incompressible solver based on the volume of fluid technique, both models are implemented in the following way: for the enthalpy-porosity, the latent heat term is appended in the energy equation which calls a function-library. The Lee model is included in the OpenFOAM framework. However, as an objective within the completion of this thesis, it is included the Lee formulation with the newly proposed term calculated through the \textit{Classical Nucleation Theory}. Both models present interface porosity models based on the Darcy's law. Results on the enthalpy-porosity model showed a trend in overpredicting the velocity magnitude and thereby, predicting a faster formation of a well-developed shape ice-front. Comparatively, Lee model with the nucleation parameters tend to underpredict the development of the ice front leading local maximal discrepancies below 33\% of error in the axial velocity displayed in the horizontal mid-plane. However, the results obtained seem to be 0.1 $mm$ shifted when compared with the Fluent results and so, the error might be slightly affected by this. Based on the experimental data found, a new geometry consisting on a plane cylinder is developed so as to see the transient evolution of the temperature, velocity magnitude and liquid fraction distributions. The results found at 5000s showed large differences in the Enthalpy-porosity model when compared with the experimental works. The evolution of the temperature for that model is telling that as with the other models, the solidification by that time is not achieved yet at an specific part of the computational domain. Initially, at lower time steps, the evolution of phyisics on this model seems to match with Lee-CNT model and with bibliographic data. However, as the solution evolves, the velocity gets reduced leading a lower ice formation.

\noindent On this part, the proposed solution for the Lee model computed through the nucleation theory is compared with the Neumann solutions of the classic Stefan problem. Initially, the numerical solution does not match with the analytical solution. This discrepancy mainly arises from the fact that the analytical solution of the classical Stefan problem exists within a limited range of idealized situations. The principal assumptions in the Stefan problem are considering constant thermophysical properties, constant solidification temperatures within phases and assuming constant the latent heat. In the work of Lee, the evolution of the latent heat is time dependent. This energy term is obtained from enthalpies of fusion between phases, temperature differences and net mass transfer, which is remarkable to say that it is mainly calculated by means of density differences and the nucleation term. And this latter term is, indeed, time dependent as well. As the last jarring term, the density is implemented so as it is function of the temperature and therefore, time dependent.

\noindent On the third part of the thesis, a proposed new solver which couples both the conduction heat transfer in the basis of a multiphase solver with the Lee-CNT and enthalpy-porosity models carried out on previous sections. The native solver is adapted so it can handle the calculation of the latent heat by means of the proposed library and it can use the proposed new equation of state. This is done as a way of comparing the obtained solutions by means of the new solver with the solutions of a native solver. The results are not the expected ones as the buoyancy effects are not reflected. The temperature, velocity magnitude and liquid fraction (ice layer evolution) seem to behave radially as if the gravity related terms were not present. Stability and control of residuals per equation are established so as to prevent unphysical behavior in the solution. Contrarily, the solution of the native solver showed similar results as with the solidification case. The slowness on the ice layer evolution now is stongly influenced by the conduction heat transfer. 

